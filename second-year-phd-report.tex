%% The first command in your LaTeX source must be the \documentclass command.
%%
%% Options:
%% twocolumn : Two column layout.
%% hf: enable header and footer.
\documentclass[
% twocolumn,
% hf,
]{ceurart}

%%
%% One can fix some overfulls
\sloppy

%\usepackage{second-year-phd-report}
\usepackage{listings}
\usepackage{tabularx}
\usepackage{subcaption}
\usepackage{myacronyms}
\usepackage[most]{tcolorbox}
%% LAST requirements!!
\usepackage{cleveref}
%% auto break lines
\lstset{breaklines=true}

%%
%% end of the preamble, start of the body of the document source.
\begin{document}

%%
%% Rights management information.
%% CC-BY is default license.
\copyrightyear{2024}
%\copyrightclause{Copyright for this paper by its authors.
%  Use permitted under Creative Commons License Attribution 4.0
%  International (CC BY 4.0).}

%%
%% This command is for the conference information
\conference{PhD course in Computer Science and Engineering, XXXIII cycle, University of Bologna, Italy}

%%
%% The "title" command
\title{Second Year PhD Report}

%%

\author[1]{Matteo Magnini}[%
orcid=0000-0001-9990-420X,
email=matteo.magnini@unibo.it,
]

\address[1]{Department of Computer Science and Engineering, University of Bologna, Via dell'Università 50, Cesena, Italy}

%%
%% The abstract is a short summary of the work to be presented in the
%% article.
%\begin{abstract}
%Lorem
%\end{abstract}

%\begin{keywords}
%\end{keywords}

\maketitle

\section{Introduction}
\label{sec:introduction}

During the second year of my PhD, I have been working on several tasks, including research publications, teaching, visiting, participation in conferences, supervision of graduating students, and PhD courses.
%
In this report, I will provide a brief -- yet exhausting -- overview of the activities I have carried out during the second year.

\section{Publications}
\label{sec:publications}

~\cite{DBLP:journals/csur/CiattoSAMO24}

\section{Period Abroad}
\label{sec:period-abroad}

I have moved to Norway for a period of 3 months, from the 5th of March to the 8th of June 2024.
%
I was hosted by Prof.~Ana Ozaki at the University of Oslo, where I had the opportunity to work with her along with her research group.
%
During this period, I have been working on the development of a new tool for the learning of ontologies by means of \ac{LLM}.
%
In particular, the work started from the ExactLearner tool\footnote{\url{https://github.com/ExactLearner/ExactLearner}}, and it has been extended to support the learning of ontologies from text\footnote{\url{https://github.com/MatteoMagnini/ExactLearner}}.
%
One paper has been already published~\cite{DBLP:conf/dlog/MagniniOS24}, and a more comprehensive work is currently undergoing.

\section{Teaching}
\label{sec:results}

During the second year of my PhD, I have been involved in the following teaching activities:
%
\begin{description}
    \item[Distributed System 2024/2025] Teaching assistant for the course of \href{https://www.unibo.it/it/studiare/dottorati-master-specializzazioni-e-altra-formazione/insegnamenti/insegnamento/2024/493397}{Distributed Systems}, held by Prof.~Andrea Omicini for a total of 24 hours.
    %
    \item[Distributed System 2023/2024] Teaching assistant for the course of \href{https://www.unibo.it/it/studiare/dottorati-master-specializzazioni-e-altra-formazione/insegnamenti/insegnamento/2023/493397}{Distributed Systems}, held by Prof.~Andrea Omicini for a total of 24 hours.
    %
\end{description}

\section{Talks}
\label{sec:talks}

\section{Supervision}
\label{sec:supervision}

\section{PhD Courses}
\label{sec:phd-courses}

This academic year, I was able to follow the following PhD courses:
\begin{description}
    \item[Multi-platform Programming for Research-Oriented Software] This course was held by Prof.~Giovanni Ciatto for a total of 10 hours (i.e., 2 CFUs \textbf{without} exam).
    %
    \item[An Introduction to Reinforcement Learning] This course was held by Prof.~Mirco Musolesi for a total of 16 hours (i.e., 3 CFUs \textbf{with} exam).
    %
    \item[Risk Assessment of Machine Learning for Cybersecurity] This course was held by Prof.~Fabio Pierazzi for a total of 20 hours (i.e., 4 CFUs \textbf{with} exam).
\end{description}

With this, I have completed reached the required number of CFUs for the PhD program, at least 18 CFUs with exams and maximum 6 CFUs without exams.
%
I have actually reached 20 CFUs with exams and 4 CFUs without exams.
%
List of the courses I have attended during the first year of my PhD:
%
\begin{description}
    \item[Comparative Introduction to Deep Learning Frameworks: TensorFlow,PyTorch and Jax] This course was held by Prof.~Mirco Musolesi for a total of 6 hours (i.e., 1 CFUs \textbf{with} exam).
    %
    \item[Low-rank Approaches for Data Analysis: Models, Numerical Methods and Applications] This course was held by Prof.~Flavia Esposito for a total of 10 hours (i.e., 2 CFUs \textbf{with} exam).
    %
    \item[How to Write and Publish a Research Paper in Computer Science and Engineering] This course was held by Prof.~Paolo Torroni for a total of 10 hours (i.e., 2 CFUs \textbf{with} exam).
    %
    \item[Data Platforms and Artificial Intelligence: Challenges and Applications] This course was held by Prof.~Matteo Francia for a total of 10 hours (i.e., 2 CFUs \textbf{without} exam).
    %
    \item[10th International Gran Canaria School] Summer School on Artificial Intelligence, Gran Canaria, Spain, 2023 for a total of 40 hours (i.e., 4 CFUs \textbf{with} exam).
    %
    \item[DevOps meets scientific research] This course was held by Prof.~Danilo Pianini for a total of 20 hours (i.e., 4 CFUs \textbf{with} exam).
    %
\end{description}

\bibliography{references}

\end{document}
